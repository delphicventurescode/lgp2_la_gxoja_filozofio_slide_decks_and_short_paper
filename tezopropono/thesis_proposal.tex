
\documentclass[12pt]{article}
\usepackage[utf8]{inputenc}
\usepackage{geometry}
\geometry{a4paper, margin=1in}
\usepackage{titlesec}
\usepackage{hyperref}
\usepackage{enumitem}
\usepackage{setspace}
\usepackage{lipsum}

\titleformat{\section}{\large\bfseries}{\thesection.}{0.5em}{}
\titleformat{\subsection}{\normalsize\bfseries}{\thesubsection.}{0.5em}{}
\setstretch{1.25}

\title{\textbf{La Gxoja Projekto: Reimagining Philosophy and History Through the Lens of Happiness}}
\author{Anand Manikutty}
\date{\today}

\begin{document}

\maketitle

\section*{Abstract}
This work introduces two interrelated theoretical frameworks: \textit{La Gxoja Filozofio}, a philosophy of happiness that draws on and extends the findings of positive psychology and global wisdom traditions; and \textit{La Gxoja Historio}, a historiographical re-reading of South Asian intellectual traditions that reframes ancient religious, philosophical, and social practices as early, culturally-rooted efforts toward sustainable well-being.

By building on the concept of CHEATs (Comprehensive Happiness-Enhancing Activities and Tasks), and incorporating tools ranging from AI-generated poetry to gamified self-improvement, this project aims not only to explore but also to apply its ideas. The argument advanced is that \textit{La Gxoja Projekto} goes beyond mere theory: it constitutes a living, evolving platform for mental wellness, historical reinterpretation, and philosophical innovation in the modern world.

\section{Introduction}
\begin{itemize}[leftmargin=*, label=--]
  \item Statement of purpose: Why a happiness-centric framework is needed now
  \item Background: Disillusionment with nihilism, stoicism, consumerism
  \item The double lens: Philosophical renewal (\textit{La Gxoja Filozofio}) + historical reinterpretation (\textit{La Gxoja Historio})
\end{itemize}

\section{Literature Review}
\subsection{Philosophical Foundations}
Stoicism, Epicureanism, Buddhism, Existentialism, Positive Psychology — where \textit{La Gxoja Filozofio} builds upon and diverges from these schools.

\subsection{Historical Frameworks}
Orientalism and misreadings of Hindu thought; Hinduism as \textit{But-Parasti}; colonial constructs of the “heathen”; South Asian responses to misinterpretation; and modern scholars who have reinterpreted these traditions.

\section{La Gxoja Filozofio: A New Framework for Flourishing}
\subsection{Core Tenets}
\begin{itemize}[leftmargin=*, label=--]
  \item Joy as foundation, not reward
  \item Active vs passive happiness
  \item CHEATs: Practical interventions to build joy daily
  \item Role of language: Why Esperanto-style clarity matters
\end{itemize}

\subsection{CHEATs (Ampleksaj Feliĉ-Plibonigaj Agadoj kaj Taskoj)}
Derived from work by Lyubomirsky, Seligman, and others. CHEATs include gratitude, kindness, physical wellness, sleep, and meaning-oriented actions.

\section{La Gxoja Historio: Reinterpreting the South Asian Past}
\subsection{Misinterpretation Across Time}
Hindus were misunderstood by Greeks, Islamic invaders, and British colonists. They were seen as idolaters and heathens.

\subsection{The Alternative Reading}
Dharma as psychological code, karma as growth, bhakti as emotional regulation, Ayurveda and Yoga as wellness practices — all early CHEATs.

\section{Implementation and Contributions}
\subsection{Projects Under La Gxoja Projekto}
Poetry books, games, lectures, platform (Fulmo), hackathon judging, guest lectures, and outreach.

\subsection{Technological Tools}
AI chatbot, curriculum, gamified happiness planner, slide decks, and visual storytelling.

\section{Contributions to Knowledge}
A happiness-centered framework combining psychology and philosophy; a reinterpretation of Indian intellectual history; and a practical set of tools for modern flourishing.

\section{Conclusion}
\textit{La Gxoja Projekto} offers an integrative approach to joy, wisdom, and memory. Theory becomes praxis, and happiness becomes not an escape from reality — but a deeper engagement with it.

\section*{Appendices}
\begin{itemize}[leftmargin=*, label=--]
  \item Slide decks (La Gxoja Filozofio, La Gxoja Historio)
  \item Fulmo screenshots, course outline
  \item Sample CHEATs calendar
  \item Chatbot interaction scripts
\end{itemize}

\section*{References}
\begin{itemize}[leftmargin=*, label=--]
  \item Lyubomirsky, S. \textit{The How of Happiness}
  \item Seligman, M. \textit{Flourish}
  \item Doniger, W. \textit{The Hindus: An Alternative History}
  \item Camus, A. \textit{The Myth of Sisyphus}
  \item Witzel, M. \textit{The Origins of the World’s Mythologies}
  \item Rubin, G. \textit{The Happiness Project}
  \item Nussbaum, M. \textit{Creating Capabilities}
\end{itemize}

\end{document}
